\chapter{Technical Replication}
\label{app:technical_replication}
This chapter explain how to run the proof of concept exploit for two of the vulnerable cable modems.
However, if this example can not be replicated successfully for a given modem, it could still be vulnerable to variations of the exploit.

\section{Requirements}
There are three elements to set up in this test: The modem to be exploited, the malicious server doing the attack, and a victim to access the malicious site.
Internet is not required during the proof of concept, however if not both the victim and attack must be on the modems lan network. The following software should be installed beforehand:

\begin{itemize}
    \item python and flask
    \item python pwntools:
    \begin{itemize}
        \item \url{http://docs.pwntools.com/en/stable/}
    \end{itemize}
    \item Web server hosting malicious javascript for F@st-3890 and TC7230: 
    \begin{itemize}
        \item \url{https://github.com/Lyrebirds/sagemcom-fast-3890-exploit/blob/master/exploit.py}
        \item \url{https://github.com/Lyrebirds/technicolor-tc7230-exploit}
    \end{itemize}
\end{itemize}

\section{Configuration}
In the static js file change the connection string on line 424 to the port and ip which your target modem uses for the spectrum analyzer. 
If the spectrum analyzer is protected by basic auth set change the username and password as well.

\subsection{Malicious Web Server}
Then, to run the http server hosting the javascript exploiting the modem, run the following command, inside the root folder containing it.
\begin{lstlisting}[numbers=none]
    sudo python exploit.py
\end{lstlisting}

\section{Replicating the Exploit}
When a victim accesses the web server the modem will be exploited. For testing it on a remote cable modem change the ip on 32 in exploit.py to the ip address to the external ip address. This can be obtained from flask upon the initial request from the victim. 
This will send a rop chain to the cable modem to be executed, and after that the modem will expect a shellcode to be send to it and executed. The F@st-3890 exploit will listen for tcp connections on port 1337 for the shellcode to be send over. The TC7230 rop chain creates a tcp connection to 194.168.22.11 for the shellcode to be send over and executed.
In both cases the exploit.py will handle sending the shellcode automatically but in case of the TC7230 be sure the attackers ip local ip is 194.168.22.11 or change it in the rop chain in the static js file.

\subsection{DNS rebind attack}
The TC7230 repository also includes a exploit using DNS rebind attack which can be run using:

\begin{lstlisting}[numbers=none]
    sudo python dns.py
\end{lstlisting}

When clicking on the link malicious javascript and html will be served and the javascript will keep trying to establish a websocket connection to the domain name. 
After the second request, the dns server will start to redirect requests against the domain name to the modem. 
At some point the browser will update the IP address associated with the domain name and the javascript will establish a http connection to the cable modem initiating the exploit.
This exploit only works for the TC7230 and will be explained in more details in the following chapter.